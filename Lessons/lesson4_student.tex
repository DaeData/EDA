\documentclass[]{article}
\usepackage{lmodern}
\usepackage{amssymb,amsmath}
\usepackage{ifxetex,ifluatex}
\usepackage{fixltx2e} % provides \textsubscript
\ifnum 0\ifxetex 1\fi\ifluatex 1\fi=0 % if pdftex
  \usepackage[T1]{fontenc}
  \usepackage[utf8]{inputenc}
\else % if luatex or xelatex
  \ifxetex
    \usepackage{mathspec}
  \else
    \usepackage{fontspec}
  \fi
  \defaultfontfeatures{Ligatures=TeX,Scale=MatchLowercase}
\fi
% use upquote if available, for straight quotes in verbatim environments
\IfFileExists{upquote.sty}{\usepackage{upquote}}{}
% use microtype if available
\IfFileExists{microtype.sty}{%
\usepackage{microtype}
\UseMicrotypeSet[protrusion]{basicmath} % disable protrusion for tt fonts
}{}
\usepackage[margin=1in]{geometry}
\usepackage{hyperref}
\hypersetup{unicode=true,
            pdfborder={0 0 0},
            breaklinks=true}
\urlstyle{same}  % don't use monospace font for urls
\usepackage{color}
\usepackage{fancyvrb}
\newcommand{\VerbBar}{|}
\newcommand{\VERB}{\Verb[commandchars=\\\{\}]}
\DefineVerbatimEnvironment{Highlighting}{Verbatim}{commandchars=\\\{\}}
% Add ',fontsize=\small' for more characters per line
\usepackage{framed}
\definecolor{shadecolor}{RGB}{248,248,248}
\newenvironment{Shaded}{\begin{snugshade}}{\end{snugshade}}
\newcommand{\KeywordTok}[1]{\textcolor[rgb]{0.13,0.29,0.53}{\textbf{#1}}}
\newcommand{\DataTypeTok}[1]{\textcolor[rgb]{0.13,0.29,0.53}{#1}}
\newcommand{\DecValTok}[1]{\textcolor[rgb]{0.00,0.00,0.81}{#1}}
\newcommand{\BaseNTok}[1]{\textcolor[rgb]{0.00,0.00,0.81}{#1}}
\newcommand{\FloatTok}[1]{\textcolor[rgb]{0.00,0.00,0.81}{#1}}
\newcommand{\ConstantTok}[1]{\textcolor[rgb]{0.00,0.00,0.00}{#1}}
\newcommand{\CharTok}[1]{\textcolor[rgb]{0.31,0.60,0.02}{#1}}
\newcommand{\SpecialCharTok}[1]{\textcolor[rgb]{0.00,0.00,0.00}{#1}}
\newcommand{\StringTok}[1]{\textcolor[rgb]{0.31,0.60,0.02}{#1}}
\newcommand{\VerbatimStringTok}[1]{\textcolor[rgb]{0.31,0.60,0.02}{#1}}
\newcommand{\SpecialStringTok}[1]{\textcolor[rgb]{0.31,0.60,0.02}{#1}}
\newcommand{\ImportTok}[1]{#1}
\newcommand{\CommentTok}[1]{\textcolor[rgb]{0.56,0.35,0.01}{\textit{#1}}}
\newcommand{\DocumentationTok}[1]{\textcolor[rgb]{0.56,0.35,0.01}{\textbf{\textit{#1}}}}
\newcommand{\AnnotationTok}[1]{\textcolor[rgb]{0.56,0.35,0.01}{\textbf{\textit{#1}}}}
\newcommand{\CommentVarTok}[1]{\textcolor[rgb]{0.56,0.35,0.01}{\textbf{\textit{#1}}}}
\newcommand{\OtherTok}[1]{\textcolor[rgb]{0.56,0.35,0.01}{#1}}
\newcommand{\FunctionTok}[1]{\textcolor[rgb]{0.00,0.00,0.00}{#1}}
\newcommand{\VariableTok}[1]{\textcolor[rgb]{0.00,0.00,0.00}{#1}}
\newcommand{\ControlFlowTok}[1]{\textcolor[rgb]{0.13,0.29,0.53}{\textbf{#1}}}
\newcommand{\OperatorTok}[1]{\textcolor[rgb]{0.81,0.36,0.00}{\textbf{#1}}}
\newcommand{\BuiltInTok}[1]{#1}
\newcommand{\ExtensionTok}[1]{#1}
\newcommand{\PreprocessorTok}[1]{\textcolor[rgb]{0.56,0.35,0.01}{\textit{#1}}}
\newcommand{\AttributeTok}[1]{\textcolor[rgb]{0.77,0.63,0.00}{#1}}
\newcommand{\RegionMarkerTok}[1]{#1}
\newcommand{\InformationTok}[1]{\textcolor[rgb]{0.56,0.35,0.01}{\textbf{\textit{#1}}}}
\newcommand{\WarningTok}[1]{\textcolor[rgb]{0.56,0.35,0.01}{\textbf{\textit{#1}}}}
\newcommand{\AlertTok}[1]{\textcolor[rgb]{0.94,0.16,0.16}{#1}}
\newcommand{\ErrorTok}[1]{\textcolor[rgb]{0.64,0.00,0.00}{\textbf{#1}}}
\newcommand{\NormalTok}[1]{#1}
\usepackage{graphicx,grffile}
\makeatletter
\def\maxwidth{\ifdim\Gin@nat@width>\linewidth\linewidth\else\Gin@nat@width\fi}
\def\maxheight{\ifdim\Gin@nat@height>\textheight\textheight\else\Gin@nat@height\fi}
\makeatother
% Scale images if necessary, so that they will not overflow the page
% margins by default, and it is still possible to overwrite the defaults
% using explicit options in \includegraphics[width, height, ...]{}
\setkeys{Gin}{width=\maxwidth,height=\maxheight,keepaspectratio}
\IfFileExists{parskip.sty}{%
\usepackage{parskip}
}{% else
\setlength{\parindent}{0pt}
\setlength{\parskip}{6pt plus 2pt minus 1pt}
}
\setlength{\emergencystretch}{3em}  % prevent overfull lines
\providecommand{\tightlist}{%
  \setlength{\itemsep}{0pt}\setlength{\parskip}{0pt}}
\setcounter{secnumdepth}{0}
% Redefines (sub)paragraphs to behave more like sections
\ifx\paragraph\undefined\else
\let\oldparagraph\paragraph
\renewcommand{\paragraph}[1]{\oldparagraph{#1}\mbox{}}
\fi
\ifx\subparagraph\undefined\else
\let\oldsubparagraph\subparagraph
\renewcommand{\subparagraph}[1]{\oldsubparagraph{#1}\mbox{}}
\fi

%%% Use protect on footnotes to avoid problems with footnotes in titles
\let\rmarkdownfootnote\footnote%
\def\footnote{\protect\rmarkdownfootnote}

%%% Change title format to be more compact
\usepackage{titling}

% Create subtitle command for use in maketitle
\newcommand{\subtitle}[1]{
  \posttitle{
    \begin{center}\large#1\end{center}
    }
}

\setlength{\droptitle}{-2em}
  \title{}
  \pretitle{\vspace{\droptitle}}
  \posttitle{}
  \author{}
  \preauthor{}\postauthor{}
  \date{}
  \predate{}\postdate{}


\begin{document}

\section{Lesson 4}\label{lesson-4}

\begin{center}\rule{0.5\linewidth}{\linethickness}\end{center}

\subsubsection{Scatterplots and Perceived Audience
Size}\label{scatterplots-and-perceived-audience-size}

Notes:

\begin{center}\rule{0.5\linewidth}{\linethickness}\end{center}

\subsubsection{Scatterplots}\label{scatterplots}

Notes: qplot(age, friend\_count, data = pf) can be used because qplot
uses x, y format. ggplot(aes(x = age, y = friend\_count), data = pf) +
geom\_point()

\begin{Shaded}
\begin{Highlighting}[]
\KeywordTok{library}\NormalTok{(ggplot2)}
\NormalTok{pf <-}\StringTok{ }\KeywordTok{read.csv}\NormalTok{(}\StringTok{'pseudo_facebook.tsv'}\NormalTok{, }\DataTypeTok{sep =} \StringTok{'}\CharTok{\textbackslash{}t}\StringTok{'}\NormalTok{)}

\KeywordTok{qplot}\NormalTok{(}\DataTypeTok{x =}\NormalTok{ age, }\DataTypeTok{y =}\NormalTok{ friend_count, }\DataTypeTok{data =}\NormalTok{ pf)}
\end{Highlighting}
\end{Shaded}

\includegraphics{lesson4_student_files/figure-latex/Scatterplots-1.pdf}

\begin{center}\rule{0.5\linewidth}{\linethickness}\end{center}

\paragraph{What are some things that you notice right
away?}\label{what-are-some-things-that-you-notice-right-away}

Response: ages extends beyond 90 and users lower than 30 have many more
friends. ***

\subsubsection{ggplot Syntax}\label{ggplot-syntax}

Notes:

\begin{Shaded}
\begin{Highlighting}[]
\KeywordTok{ggplot}\NormalTok{(}\KeywordTok{aes}\NormalTok{(}\DataTypeTok{x =}\NormalTok{ age, }\DataTypeTok{y =}\NormalTok{ friend_count), }\DataTypeTok{data =}\NormalTok{ pf) }\OperatorTok{+}
\StringTok{  }\KeywordTok{geom_point}\NormalTok{() }\OperatorTok{+}\StringTok{ }\KeywordTok{xlim}\NormalTok{(}\DecValTok{30}\NormalTok{, }\DecValTok{90}\NormalTok{)}
\end{Highlighting}
\end{Shaded}

\begin{verbatim}
## Warning: Removed 56588 rows containing missing values (geom_point).
\end{verbatim}

\includegraphics{lesson4_student_files/figure-latex/ggplot Syntax-1.pdf}

\begin{center}\rule{0.5\linewidth}{\linethickness}\end{center}

\subsubsection{Overplotting}\label{overplotting}

Notes: age is a continuous value. It is expressed as an int in the plot
and as such makes the columns line up neatly which is not a natural
state. By adding jitter we introduce noise to the data giving a more
realistic reflection of the data dispersion.

\begin{Shaded}
\begin{Highlighting}[]
\KeywordTok{ggplot}\NormalTok{(}\KeywordTok{aes}\NormalTok{(}\DataTypeTok{x =}\NormalTok{ age, }\DataTypeTok{y =}\NormalTok{ friend_count), }\DataTypeTok{data =}\NormalTok{ pf) }\OperatorTok{+}
\StringTok{  }\KeywordTok{geom_jitter}\NormalTok{(}\DataTypeTok{alpha =} \DecValTok{1}\OperatorTok{/}\DecValTok{20}\NormalTok{) }\OperatorTok{+}\StringTok{ }
\StringTok{  }\KeywordTok{xlim}\NormalTok{(}\DecValTok{30}\NormalTok{, }\DecValTok{90}\NormalTok{) }
\end{Highlighting}
\end{Shaded}

\begin{verbatim}
## Warning: Removed 57478 rows containing missing values (geom_point).
\end{verbatim}

\includegraphics{lesson4_student_files/figure-latex/Overplotting-1.pdf}

\paragraph{What do you notice in the
plot?}\label{what-do-you-notice-in-the-plot}

Response: The information scatter feels more realistic and a truer
reflection of how the data would be represented. It is also clearer that
the age value of 69 is inaccurate. Younger users do not seem to be as
high as they were before. ***

\subsubsection{Coord\_trans()}\label{coord_trans}

Notes:

\begin{Shaded}
\begin{Highlighting}[]
\KeywordTok{ggplot}\NormalTok{(}\KeywordTok{aes}\NormalTok{(}\DataTypeTok{x =}\NormalTok{ age, }\DataTypeTok{y =}\NormalTok{ friend_count), }\DataTypeTok{data =}\NormalTok{ pf) }\OperatorTok{+}
\StringTok{  }\KeywordTok{geom_jitter}\NormalTok{(}\DataTypeTok{alpha =} \DecValTok{1}\OperatorTok{/}\DecValTok{20}\NormalTok{) }\OperatorTok{+}\StringTok{ }
\StringTok{  }\KeywordTok{xlim}\NormalTok{(}\DecValTok{30}\NormalTok{, }\DecValTok{90}\NormalTok{)}
\end{Highlighting}
\end{Shaded}

\begin{verbatim}
## Warning: Removed 57500 rows containing missing values (geom_point).
\end{verbatim}

\includegraphics{lesson4_student_files/figure-latex/Coord_trans()-1.pdf}

\paragraph{Look up the documentation for coord\_trans() and add a layer
to the plot that transforms friend\_count using the square root
function. Create your
plot!}\label{look-up-the-documentation-for-coord_trans-and-add-a-layer-to-the-plot-that-transforms-friend_count-using-the-square-root-function.-create-your-plot}

\begin{Shaded}
\begin{Highlighting}[]
\KeywordTok{ggplot}\NormalTok{(}\KeywordTok{aes}\NormalTok{(}\DataTypeTok{x =}\NormalTok{ age, }\DataTypeTok{y =}\NormalTok{ friend_count), }\DataTypeTok{data =}\NormalTok{ pf) }\OperatorTok{+}
\StringTok{  }\KeywordTok{geom_point}\NormalTok{(}\DataTypeTok{alpha =} \DecValTok{1}\OperatorTok{/}\DecValTok{20}\NormalTok{) }\OperatorTok{+}\StringTok{ }
\StringTok{  }\KeywordTok{xlim}\NormalTok{(}\DecValTok{30}\NormalTok{, }\DecValTok{90}\NormalTok{) }\OperatorTok{+}
\StringTok{  }\KeywordTok{coord_trans}\NormalTok{(}\DataTypeTok{y =} \StringTok{'sqrt'}\NormalTok{)}
\end{Highlighting}
\end{Shaded}

\begin{verbatim}
## Warning: Removed 56588 rows containing missing values (geom_point).
\end{verbatim}

\includegraphics{lesson4_student_files/figure-latex/unnamed-chunk-1-1.pdf}

\paragraph{What do you notice?}\label{what-do-you-notice}

By adding the sqrt transform it zooms in on the datapoints and we can
see that actual count concentration is well below 1000. ***

\subsubsection{Alpha and Jitter}\label{alpha-and-jitter}

Notes:

\begin{Shaded}
\begin{Highlighting}[]
\KeywordTok{ggplot}\NormalTok{(}\KeywordTok{aes}\NormalTok{(}\DataTypeTok{x =}\NormalTok{ age, }\DataTypeTok{y =}\NormalTok{ friendships_initiated), }\DataTypeTok{data =}\NormalTok{ pf) }\OperatorTok{+}\StringTok{ }
\StringTok{  }\KeywordTok{geom_jitter}\NormalTok{(}\DataTypeTok{alpha =} \DecValTok{1}\OperatorTok{/}\DecValTok{10}\NormalTok{, }\DataTypeTok{position =} \KeywordTok{position_jitter}\NormalTok{(}\DataTypeTok{h =} \DecValTok{0}\NormalTok{)) }\OperatorTok{+}
\StringTok{  }\KeywordTok{coord_trans}\NormalTok{(}\DataTypeTok{y =} \StringTok{'sqrt'}\NormalTok{)}
\end{Highlighting}
\end{Shaded}

\includegraphics{lesson4_student_files/figure-latex/Alpha and Jitter-1.pdf}

\begin{center}\rule{0.5\linewidth}{\linethickness}\end{center}

\subsubsection{Overplotting and Domain
Knowledge}\label{overplotting-and-domain-knowledge}

Notes:

\begin{center}\rule{0.5\linewidth}{\linethickness}\end{center}

\subsubsection{Conditional Means}\label{conditional-means}

Notes:

\begin{Shaded}
\begin{Highlighting}[]
\KeywordTok{library}\NormalTok{(dplyr)}
\end{Highlighting}
\end{Shaded}

\begin{verbatim}
## 
## Attaching package: 'dplyr'
\end{verbatim}

\begin{verbatim}
## The following objects are masked from 'package:stats':
## 
##     filter, lag
\end{verbatim}

\begin{verbatim}
## The following objects are masked from 'package:base':
## 
##     intersect, setdiff, setequal, union
\end{verbatim}

\begin{Shaded}
\begin{Highlighting}[]
\NormalTok{age_groups <-}\StringTok{ }\KeywordTok{group_by}\NormalTok{(pf, age)}

\NormalTok{pf.fc_by_age <-}\StringTok{ }\KeywordTok{summarise}\NormalTok{(age_groups,}
                          \DataTypeTok{friend_count_mean =} \KeywordTok{mean}\NormalTok{(friend_count),}
                          \DataTypeTok{friend_count_median =} \KeywordTok{median}\NormalTok{(friend_count),}
                          \DataTypeTok{n =} \KeywordTok{n}\NormalTok{())}
\NormalTok{pf.fc_by_age <-}\StringTok{ }\KeywordTok{arrange}\NormalTok{(pf.fc_by_age, age)}
\KeywordTok{head}\NormalTok{(pf.fc_by_age)}
\end{Highlighting}
\end{Shaded}

\begin{verbatim}
## # A tibble: 6 x 4
##     age friend_count_mean friend_count_median     n
##   <int>             <dbl>               <dbl> <int>
## 1    13              165.                 74    484
## 2    14              251.                132   1925
## 3    15              348.                161   2618
## 4    16              352.                172.  3086
## 5    17              350.                156   3283
## 6    18              331.                162   5196
\end{verbatim}

\begin{Shaded}
\begin{Highlighting}[]
\NormalTok{pf }\OperatorTok
\StringTok{  }\KeywordTok{group_by}\NormalTok{(age) }\OperatorTok
\StringTok{  }\KeywordTok{summarise}\NormalTok{(}\DataTypeTok{friend_count_mean =} \KeywordTok{mean}\NormalTok{(friend_count),}
            \DataTypeTok{friend_count_median =} \KeywordTok{median}\NormalTok{(friend_count),}
            \DataTypeTok{n =} \KeywordTok{n}\NormalTok{()) }\OperatorTok
\StringTok{  }\KeywordTok{arrange}\NormalTok{(age)}
\end{Highlighting}
\end{Shaded}

\begin{verbatim}
## # A tibble: 101 x 4
##      age friend_count_mean friend_count_median     n
##    <int>             <dbl>               <dbl> <int>
##  1    13              165.                 74    484
##  2    14              251.                132   1925
##  3    15              348.                161   2618
##  4    16              352.                172.  3086
##  5    17              350.                156   3283
##  6    18              331.                162   5196
##  7    19              334.                157   4391
##  8    20              283.                135   3769
##  9    21              236.                121   3671
## 10    22              211.                106   3032
## # ... with 91 more rows
\end{verbatim}

\begin{Shaded}
\begin{Highlighting}[]
\KeywordTok{head}\NormalTok{(pf.fc_by_age, }\DecValTok{20}\NormalTok{)}
\end{Highlighting}
\end{Shaded}

\begin{verbatim}
## # A tibble: 20 x 4
##      age friend_count_mean friend_count_median     n
##    <int>             <dbl>               <dbl> <int>
##  1    13              165.                74     484
##  2    14              251.               132    1925
##  3    15              348.               161    2618
##  4    16              352.               172.   3086
##  5    17              350.               156    3283
##  6    18              331.               162    5196
##  7    19              334.               157    4391
##  8    20              283.               135    3769
##  9    21              236.               121    3671
## 10    22              211.               106    3032
## 11    23              203.                93    4404
## 12    24              186.                92    2827
## 13    25              131.                62    3641
## 14    26              144.                75    2815
## 15    27              134.                72    2240
## 16    28              126.                66    2364
## 17    29              121.                66    1936
## 18    30              115.                67.5  1716
## 19    31              118.                63    1694
## 20    32              114.                63    1443
\end{verbatim}

Create your plot!

\begin{Shaded}
\begin{Highlighting}[]
\KeywordTok{ggplot}\NormalTok{(}\KeywordTok{aes}\NormalTok{(age, friend_count_mean),}\DataTypeTok{data =}\NormalTok{ pf.fc_by_age) }\OperatorTok{+}
\StringTok{  }\KeywordTok{geom_line}\NormalTok{()}
\end{Highlighting}
\end{Shaded}

\includegraphics{lesson4_student_files/figure-latex/Conditional Means Plot-1.pdf}

\begin{center}\rule{0.5\linewidth}{\linethickness}\end{center}

\subsubsection{Overlaying Summaries with Raw
Data}\label{overlaying-summaries-with-raw-data}

Notes: fun.y take a function and applies it

\begin{Shaded}
\begin{Highlighting}[]
\KeywordTok{ggplot}\NormalTok{(}\KeywordTok{aes}\NormalTok{(}\DataTypeTok{x =}\NormalTok{ age, }\DataTypeTok{y =}\NormalTok{ friend_count), }\DataTypeTok{data =}\NormalTok{ pf) }\OperatorTok{+}
\StringTok{  }\KeywordTok{coord_cartesian}\NormalTok{(}\DataTypeTok{xlim =} \KeywordTok{c}\NormalTok{(}\DecValTok{13}\NormalTok{,}\DecValTok{90}\NormalTok{)) }\OperatorTok{+}
\StringTok{  }\KeywordTok{geom_point}\NormalTok{(}\DataTypeTok{alpha =} \FloatTok{0.05}\NormalTok{,}
             \DataTypeTok{position =} \KeywordTok{position_jitter}\NormalTok{(}\DataTypeTok{h =} \DecValTok{0}\NormalTok{),}
             \DataTypeTok{color =} \StringTok{'orange'}\NormalTok{) }\OperatorTok{+}
\StringTok{  }\KeywordTok{coord_trans}\NormalTok{(}\DataTypeTok{y =} \StringTok{'sqrt'}\NormalTok{) }\OperatorTok{+}\StringTok{ }
\StringTok{  }\KeywordTok{geom_line}\NormalTok{(}\DataTypeTok{stat =} \StringTok{'summary'}\NormalTok{, }\DataTypeTok{fun.y =}\NormalTok{ mean) }\OperatorTok{+}
\StringTok{  }\KeywordTok{geom_line}\NormalTok{(}\DataTypeTok{stat =} \StringTok{'summary'}\NormalTok{, }\DataTypeTok{fun.y =}\NormalTok{ quantile, }\DataTypeTok{fun.args =} \KeywordTok{list}\NormalTok{(}\DataTypeTok{probs =}\NormalTok{ .}\DecValTok{1}\NormalTok{),}
            \DataTypeTok{linetype =} \DecValTok{2}\NormalTok{, }\DataTypeTok{color =} \StringTok{'blue'}\NormalTok{) }\OperatorTok{+}
\StringTok{  }\KeywordTok{geom_line}\NormalTok{(}\DataTypeTok{stat =} \StringTok{'summary'}\NormalTok{, }\DataTypeTok{fun.y =}\NormalTok{ quantile, }\DataTypeTok{fun.args =} \KeywordTok{list}\NormalTok{(}\DataTypeTok{probs =}\NormalTok{ .}\DecValTok{5}\NormalTok{),}
            \DataTypeTok{color =} \StringTok{'blue'}\NormalTok{) }\OperatorTok{+}
\StringTok{  }\KeywordTok{geom_line}\NormalTok{(}\DataTypeTok{stat =} \StringTok{'summary'}\NormalTok{, }\DataTypeTok{fun.y =}\NormalTok{ quantile, }\DataTypeTok{fun.args =} \KeywordTok{list}\NormalTok{(}\DataTypeTok{probs =}\NormalTok{ .}\DecValTok{9}\NormalTok{),}
            \DataTypeTok{linetype =} \DecValTok{2}\NormalTok{, }\DataTypeTok{color =} \StringTok{'blue'}\NormalTok{)}
\end{Highlighting}
\end{Shaded}

\includegraphics{lesson4_student_files/figure-latex/Overlaying Summaries with Raw Data-1.pdf}

\paragraph{What are some of your observations of the
plot?}\label{what-are-some-of-your-observations-of-the-plot}

Response:

\begin{center}\rule{0.5\linewidth}{\linethickness}\end{center}

\subsubsection{Moira: Histogram Summary and
Scatterplot}\label{moira-histogram-summary-and-scatterplot}

See the Instructor Notes of this video to download Moira's paper on
perceived audience size and to see the final plot.

Notes:

\begin{center}\rule{0.5\linewidth}{\linethickness}\end{center}

\subsubsection{Correlation}\label{correlation}

Notes:

\begin{Shaded}
\begin{Highlighting}[]
\KeywordTok{cor.test}\NormalTok{(pf}\OperatorTok{$}\NormalTok{age, pf}\OperatorTok{$}\NormalTok{friend_count, }\DataTypeTok{method =} \StringTok{'pearson'}\NormalTok{)}
\end{Highlighting}
\end{Shaded}

\begin{verbatim}
## 
##  Pearson's product-moment correlation
## 
## data:  pf$age and pf$friend_count
## t = -8.6268, df = 99001, p-value < 2.2e-16
## alternative hypothesis: true correlation is not equal to 0
## 95 percent confidence interval:
##  -0.03363072 -0.02118189
## sample estimates:
##         cor 
## -0.02740737
\end{verbatim}

\begin{Shaded}
\begin{Highlighting}[]
\KeywordTok{with}\NormalTok{(pf, }\KeywordTok{cor.test}\NormalTok{(age, friend_count, }\DataTypeTok{method =} \StringTok{'pearson'}\NormalTok{))}
\end{Highlighting}
\end{Shaded}

\begin{verbatim}
## 
##  Pearson's product-moment correlation
## 
## data:  age and friend_count
## t = -8.6268, df = 99001, p-value < 2.2e-16
## alternative hypothesis: true correlation is not equal to 0
## 95 percent confidence interval:
##  -0.03363072 -0.02118189
## sample estimates:
##         cor 
## -0.02740737
\end{verbatim}

Look up the documentation for the cor.test function.

What's the correlation between age and friend count? Round to three
decimal places. Response: -0.02740737

\begin{center}\rule{0.5\linewidth}{\linethickness}\end{center}

\subsubsection{Correlation on Subsets}\label{correlation-on-subsets}

Notes:

\begin{Shaded}
\begin{Highlighting}[]
\KeywordTok{with}\NormalTok{(}\KeywordTok{subset}\NormalTok{(pf, age }\OperatorTok{<=}\StringTok{ }\DecValTok{70}\NormalTok{), }\KeywordTok{cor.test}\NormalTok{(age, friend_count,}
                                     \DataTypeTok{method =} \StringTok{'pearson'}\NormalTok{))}
\end{Highlighting}
\end{Shaded}

\begin{verbatim}
## 
##  Pearson's product-moment correlation
## 
## data:  age and friend_count
## t = -52.592, df = 91029, p-value < 2.2e-16
## alternative hypothesis: true correlation is not equal to 0
## 95 percent confidence interval:
##  -0.1780220 -0.1654129
## sample estimates:
##        cor 
## -0.1717245
\end{verbatim}

\begin{center}\rule{0.5\linewidth}{\linethickness}\end{center}

\subsubsection{Correlation Methods}\label{correlation-methods}

Notes:

\begin{center}\rule{0.5\linewidth}{\linethickness}\end{center}

\subsection{Create Scatterplots}\label{create-scatterplots}

Notes:

\begin{Shaded}
\begin{Highlighting}[]
\KeywordTok{ggplot}\NormalTok{(}\KeywordTok{aes}\NormalTok{(www_likes_received, }\DataTypeTok{y =}\NormalTok{ likes_received), }\DataTypeTok{data =}\NormalTok{ pf) }\OperatorTok{+}
\StringTok{  }\KeywordTok{geom_point}\NormalTok{()}
\end{Highlighting}
\end{Shaded}

\includegraphics{lesson4_student_files/figure-latex/unnamed-chunk-2-1.pdf}

\begin{center}\rule{0.5\linewidth}{\linethickness}\end{center}

\subsubsection{Strong Correlations}\label{strong-correlations}

Notes:

\begin{Shaded}
\begin{Highlighting}[]
\KeywordTok{ggplot}\NormalTok{(}\KeywordTok{aes}\NormalTok{(www_likes_received, }\DataTypeTok{y =}\NormalTok{ likes_received), }\DataTypeTok{data =}\NormalTok{ pf) }\OperatorTok{+}
\StringTok{  }\KeywordTok{geom_point}\NormalTok{() }\OperatorTok{+}
\StringTok{  }\KeywordTok{xlim}\NormalTok{(}\DecValTok{0}\NormalTok{, }\KeywordTok{quantile}\NormalTok{(pf}\OperatorTok{$}\NormalTok{www_likes_received, }\FloatTok{0.95}\NormalTok{)) }\OperatorTok{+}
\StringTok{  }\KeywordTok{ylim}\NormalTok{(}\DecValTok{0}\NormalTok{, }\KeywordTok{quantile}\NormalTok{(pf}\OperatorTok{$}\NormalTok{likes_received, }\FloatTok{0.95}\NormalTok{)) }\OperatorTok{+}
\StringTok{  }\KeywordTok{geom_smooth}\NormalTok{(}\DataTypeTok{method =} \StringTok{'lm'}\NormalTok{, }\DataTypeTok{color =} \StringTok{'purple'}\NormalTok{)}
\end{Highlighting}
\end{Shaded}

\begin{verbatim}
## Warning: Removed 6075 rows containing non-finite values (stat_smooth).
\end{verbatim}

\begin{verbatim}
## Warning: Removed 6075 rows containing missing values (geom_point).
\end{verbatim}

\includegraphics{lesson4_student_files/figure-latex/Strong Correlations-1.pdf}

What's the correlation betwen the two variables? Include the top 5\% of
values for the variable in the calculation and round to 3 decimal
places.

\begin{Shaded}
\begin{Highlighting}[]
\KeywordTok{with}\NormalTok{(pf, }\KeywordTok{cor.test}\NormalTok{(www_likes_received, likes_received, }\DataTypeTok{method =} \StringTok{'pearson'}\NormalTok{))}
\end{Highlighting}
\end{Shaded}

\begin{verbatim}
## 
##  Pearson's product-moment correlation
## 
## data:  www_likes_received and likes_received
## t = 937.1, df = 99001, p-value < 2.2e-16
## alternative hypothesis: true correlation is not equal to 0
## 95 percent confidence interval:
##  0.9473553 0.9486176
## sample estimates:
##       cor 
## 0.9479902
\end{verbatim}

Response:

\begin{center}\rule{0.5\linewidth}{\linethickness}\end{center}

\subsubsection{Moira on Correlation}\label{moira-on-correlation}

Notes:

\begin{center}\rule{0.5\linewidth}{\linethickness}\end{center}

\subsubsection{More Caution with
Correlation}\label{more-caution-with-correlation}

Notes:

\begin{Shaded}
\begin{Highlighting}[]
\KeywordTok{library}\NormalTok{(alr3)}
\end{Highlighting}
\end{Shaded}

\begin{verbatim}
## Loading required package: car
\end{verbatim}

\begin{verbatim}
## Loading required package: carData
\end{verbatim}

\begin{verbatim}
## 
## Attaching package: 'car'
\end{verbatim}

\begin{verbatim}
## The following object is masked from 'package:dplyr':
## 
##     recode
\end{verbatim}

\begin{Shaded}
\begin{Highlighting}[]
\KeywordTok{data}\NormalTok{(}\StringTok{"Mitchell"}\NormalTok{)}
\NormalTok{?Mitchell}
\end{Highlighting}
\end{Shaded}

\begin{verbatim}
## starting httpd help server ...
\end{verbatim}

\begin{verbatim}
##  done
\end{verbatim}

\begin{Shaded}
\begin{Highlighting}[]
\KeywordTok{str}\NormalTok{(Mitchell)}
\end{Highlighting}
\end{Shaded}

\begin{verbatim}
## 'data.frame':    204 obs. of  2 variables:
##  $ Month: int  0 1 2 3 4 5 6 7 8 9 ...
##  $ Temp : num  -5.18 -1.65 2.49 10.4 14.99 ...
\end{verbatim}

\begin{Shaded}
\begin{Highlighting}[]
\KeywordTok{ggplot}\NormalTok{(}\KeywordTok{aes}\NormalTok{(}\DataTypeTok{x =}\NormalTok{ Month, }\DataTypeTok{y =}\NormalTok{ Temp), }\DataTypeTok{data =}\NormalTok{ Mitchell) }\OperatorTok{+}
\StringTok{  }\KeywordTok{geom_point}\NormalTok{()}
\end{Highlighting}
\end{Shaded}

\includegraphics{lesson4_student_files/figure-latex/More Caution With Correlation-1.pdf}

Create your plot!

\begin{Shaded}
\begin{Highlighting}[]
\KeywordTok{range}\NormalTok{(Mitchell}\OperatorTok{$}\NormalTok{Month)}
\end{Highlighting}
\end{Shaded}

\begin{verbatim}
## [1]   0 203
\end{verbatim}

\begin{Shaded}
\begin{Highlighting}[]
\KeywordTok{cor.test}\NormalTok{(Mitchell}\OperatorTok{$}\NormalTok{Month, Mitchell}\OperatorTok{$}\NormalTok{Temp, }\DataTypeTok{method =} \StringTok{'pearson'}\NormalTok{)}
\end{Highlighting}
\end{Shaded}

\begin{verbatim}
## 
##  Pearson's product-moment correlation
## 
## data:  Mitchell$Month and Mitchell$Temp
## t = 0.81816, df = 202, p-value = 0.4142
## alternative hypothesis: true correlation is not equal to 0
## 95 percent confidence interval:
##  -0.08053637  0.19331562
## sample estimates:
##        cor 
## 0.05747063
\end{verbatim}

\begin{Shaded}
\begin{Highlighting}[]
\KeywordTok{ggplot}\NormalTok{(}\KeywordTok{aes}\NormalTok{(}\DataTypeTok{x =}\NormalTok{ Month, }\DataTypeTok{y =}\NormalTok{ Temp), }\DataTypeTok{data =}\NormalTok{ Mitchell) }\OperatorTok{+}
\StringTok{  }\KeywordTok{geom_point}\NormalTok{()}
\end{Highlighting}
\end{Shaded}

\includegraphics{lesson4_student_files/figure-latex/Temp vs Month-1.pdf}

\begin{center}\rule{0.5\linewidth}{\linethickness}\end{center}

\subsubsection{Noisy Scatterplots}\label{noisy-scatterplots}

\begin{enumerate}
\def\labelenumi{\alph{enumi}.}
\item
  Take a guess for the correlation coefficient for the scatterplot.
\item
  What is the actual correlation of the two variables? (Round to the
  thousandths place)
\end{enumerate}

\begin{center}\rule{0.5\linewidth}{\linethickness}\end{center}

\subsubsection{Making Sense of Data}\label{making-sense-of-data}

Notes:

\begin{Shaded}
\begin{Highlighting}[]
\KeywordTok{range}\NormalTok{(Mitchell}\OperatorTok{$}\NormalTok{Month)}
\end{Highlighting}
\end{Shaded}

\begin{verbatim}
## [1]   0 203
\end{verbatim}

\begin{Shaded}
\begin{Highlighting}[]
\KeywordTok{cor.test}\NormalTok{(Mitchell}\OperatorTok{$}\NormalTok{Month, Mitchell}\OperatorTok{$}\NormalTok{Temp, }\DataTypeTok{method =} \StringTok{'pearson'}\NormalTok{)}
\end{Highlighting}
\end{Shaded}

\begin{verbatim}
## 
##  Pearson's product-moment correlation
## 
## data:  Mitchell$Month and Mitchell$Temp
## t = 0.81816, df = 202, p-value = 0.4142
## alternative hypothesis: true correlation is not equal to 0
## 95 percent confidence interval:
##  -0.08053637  0.19331562
## sample estimates:
##        cor 
## 0.05747063
\end{verbatim}

\begin{Shaded}
\begin{Highlighting}[]
\KeywordTok{ggplot}\NormalTok{(}\KeywordTok{aes}\NormalTok{(}\DataTypeTok{x =}\NormalTok{ Month, }\DataTypeTok{y =}\NormalTok{ Temp), }\DataTypeTok{data =}\NormalTok{ Mitchell) }\OperatorTok{+}
\StringTok{  }\KeywordTok{geom_point}\NormalTok{() }\OperatorTok{+}
\StringTok{  }\KeywordTok{scale_x_continuous}\NormalTok{(}\DataTypeTok{breaks =} \KeywordTok{seq}\NormalTok{(}\DecValTok{0}\NormalTok{, }\DecValTok{203}\NormalTok{, }\DecValTok{12}\NormalTok{))}
\end{Highlighting}
\end{Shaded}

\includegraphics{lesson4_student_files/figure-latex/Making Sense of Data-1.pdf}

\begin{center}\rule{0.5\linewidth}{\linethickness}\end{center}

\subsubsection{A New Perspective}\label{a-new-perspective}

What do you notice? Response:

Watch the solution video and check out the Instructor Notes! Notes:

\begin{center}\rule{0.5\linewidth}{\linethickness}\end{center}

\subsubsection{Understanding Noise: Age to Age
Months}\label{understanding-noise-age-to-age-months}

Notes:

\begin{Shaded}
\begin{Highlighting}[]
\NormalTok{pf}\OperatorTok{$}\NormalTok{age_with_months <-}\StringTok{ }\NormalTok{pf}\OperatorTok{$}\NormalTok{age }\OperatorTok{+}\StringTok{ }\NormalTok{(}\DecValTok{12} \OperatorTok{-}\StringTok{ }\NormalTok{pf}\OperatorTok{$}\NormalTok{dob_month) }\OperatorTok{/}\StringTok{ }\DecValTok{12}
\end{Highlighting}
\end{Shaded}

\begin{center}\rule{0.5\linewidth}{\linethickness}\end{center}

\subsubsection{Age with Months Means}\label{age-with-months-means}

\begin{Shaded}
\begin{Highlighting}[]
\NormalTok{pf.fc_by_age_months <-}\StringTok{ }\NormalTok{pf }\OperatorTok
\StringTok{  }\KeywordTok{group_by}\NormalTok{(age_with_months) }\OperatorTok
\StringTok{  }\KeywordTok{summarise}\NormalTok{(}\DataTypeTok{friend_count_mean =} \KeywordTok{mean}\NormalTok{(friend_count),}
            \DataTypeTok{friend_count_median =} \KeywordTok{median}\NormalTok{(friend_count),}
            \DataTypeTok{n =} \KeywordTok{n}\NormalTok{())}\OperatorTok
\StringTok{  }\KeywordTok{arrange}\NormalTok{(age_with_months)}
\end{Highlighting}
\end{Shaded}

Programming Assignment

\begin{Shaded}
\begin{Highlighting}[]
\KeywordTok{library}\NormalTok{(dplyr)}
\NormalTok{pf.age_with_months <-}\StringTok{ }\NormalTok{pf }\OperatorTok
\StringTok{  }\KeywordTok{group_by}\NormalTok{(age_with_months) }\OperatorTok
\StringTok{  }\KeywordTok{summarise}\NormalTok{(}\DataTypeTok{friend_count_mean =} \KeywordTok{mean}\NormalTok{(friend_count),}
            \DataTypeTok{friend_count_median =} \KeywordTok{median}\NormalTok{(friend_count),}
            \DataTypeTok{n =} \KeywordTok{n}\NormalTok{()) }\OperatorTok
\StringTok{  }\KeywordTok{arrange}\NormalTok{(age_with_months)}
\end{Highlighting}
\end{Shaded}

\begin{center}\rule{0.5\linewidth}{\linethickness}\end{center}

\subsubsection{Noise in Conditional
Means}\label{noise-in-conditional-means}

\begin{Shaded}
\begin{Highlighting}[]
\KeywordTok{ggplot}\NormalTok{(}\KeywordTok{aes}\NormalTok{(}\DataTypeTok{x =}\NormalTok{ age_with_months, }\DataTypeTok{y =}\NormalTok{ friend_count_mean),}
       \DataTypeTok{data =} \KeywordTok{subset}\NormalTok{(pf.age_with_months, age_with_months }\OperatorTok{<}\StringTok{ }\DecValTok{71}\NormalTok{)) }\OperatorTok{+}
\StringTok{  }\KeywordTok{geom_line}\NormalTok{()}
\end{Highlighting}
\end{Shaded}

\includegraphics{lesson4_student_files/figure-latex/Noise in Conditional Means-1.pdf}

\begin{center}\rule{0.5\linewidth}{\linethickness}\end{center}

\subsubsection{Smoothing Conditional
Means}\label{smoothing-conditional-means}

Notes: Bias variant tradeoff.

\begin{Shaded}
\begin{Highlighting}[]
\NormalTok{p1 <-}\StringTok{ }\KeywordTok{ggplot}\NormalTok{(}\KeywordTok{aes}\NormalTok{(}\DataTypeTok{x =}\NormalTok{ age, }\DataTypeTok{y =}\NormalTok{ friend_count_mean),}
       \DataTypeTok{data =}\NormalTok{ pf.fc_by_age) }\OperatorTok{+}
\StringTok{  }\KeywordTok{geom_line}\NormalTok{()}\OperatorTok{+}
\StringTok{  }\KeywordTok{geom_smooth}\NormalTok{()}

\NormalTok{p2 <-}\StringTok{ }\KeywordTok{ggplot}\NormalTok{(}\KeywordTok{aes}\NormalTok{(}\DataTypeTok{x =}\NormalTok{ age_with_months, }\DataTypeTok{y =}\NormalTok{ friend_count_mean),}
       \DataTypeTok{data =} \KeywordTok{subset}\NormalTok{(pf.fc_by_age_months, age_with_months }\OperatorTok{<}\StringTok{ }\DecValTok{71}\NormalTok{)) }\OperatorTok{+}
\StringTok{        }\KeywordTok{geom_line}\NormalTok{() }\OperatorTok{+}
\StringTok{        }\KeywordTok{geom_smooth}\NormalTok{()}

\NormalTok{p3 <-}\StringTok{ }\KeywordTok{ggplot}\NormalTok{(}\KeywordTok{aes}\NormalTok{(}\DataTypeTok{x =} \KeywordTok{round}\NormalTok{(age }\OperatorTok{/}\StringTok{ }\DecValTok{5}\NormalTok{) }\OperatorTok{*}\StringTok{ }\DecValTok{5}\NormalTok{, }\DataTypeTok{y =}\NormalTok{ friend_count),}
             \DataTypeTok{data =} \KeywordTok{subset}\NormalTok{(pf, age }\OperatorTok{<}\StringTok{ }\DecValTok{71}\NormalTok{)) }\OperatorTok{+}
\StringTok{  }\KeywordTok{geom_line}\NormalTok{(}\DataTypeTok{stat =} \StringTok{'summary'}\NormalTok{, }\DataTypeTok{fun.y =}\NormalTok{ mean)}
\KeywordTok{library}\NormalTok{(gridExtra)}
\end{Highlighting}
\end{Shaded}

\begin{verbatim}
## 
## Attaching package: 'gridExtra'
\end{verbatim}

\begin{verbatim}
## The following object is masked from 'package:dplyr':
## 
##     combine
\end{verbatim}

\begin{Shaded}
\begin{Highlighting}[]
\KeywordTok{grid.arrange}\NormalTok{(p1, p2, p3, }\DataTypeTok{ncol =}  \DecValTok{1}\NormalTok{)}
\end{Highlighting}
\end{Shaded}

\begin{verbatim}
## `geom_smooth()` using method = 'loess'
\end{verbatim}

\begin{verbatim}
## `geom_smooth()` using method = 'loess'
\end{verbatim}

\includegraphics{lesson4_student_files/figure-latex/Smoothing Conditional Means-1.pdf}

\begin{center}\rule{0.5\linewidth}{\linethickness}\end{center}

\subsubsection{Which Plot to Choose?}\label{which-plot-to-choose}

Notes: You don't have to choose. We will often create multiple plots and
summaries of the data. They can reveal different things about the data.
When publishing you may want to narrow the scope to the plots that best
communicate the findings. ***

\subsubsection{Analyzing Two Variables}\label{analyzing-two-variables}

Reflection:

\begin{center}\rule{0.5\linewidth}{\linethickness}\end{center}

Click \textbf{KnitHTML} to see all of your hard work and to have an html
page of this lesson, your answers, and your notes!


\end{document}
